%% stripped down version of "bare_jrnl.tex" for use in Casey's Circuits I class.
%% Original version has very good comments for use. You should check it out at
%% http://www.ieee.org/publications_standards/publications/authors/authors_journals.html
%% I run GNU/Linux so I downloaded the "Unix LaTeX2e Transactions Style File" package
%% and based my work off of the sample tex file named "bare_jrnl.tex".

% original author info below (this guy's a rockstar for making his comments so easy to use :P)
%% 2007/01/11
%% by Michael Shell
%% see http://www.michaelshell.org/
%% for current contact information.

\documentclass[journal,twocolumn]{IEEEtran}
% make sure "IEEEtran.cls" is in the path of the tex file you are working on

%graphics package for adding images
\ifCLASSINFOpdf
  \usepackage[pdftex]{graphicx}
\else
   \usepackage[dvips]{graphicx}
\fi

%math package for math equations
\usepackage[cmex10]{amsmath}

%float package for putting images where I fucking tell them to go
\usepackage{float}

%for sourcecode
\usepackage{listings}
\lstset{breaklines=true,language=vhdl,basicstyle=\scriptsize,showspaces=false,showstringspaces=false}

%For hyperlinks
\usepackage{hyperref}

% There are TONS of other packages you can use for various different cases.
% Bibliographys are big in research papers but not important for lab reports
% (I think...?). So I haven't included any bibliography code

\begin{document}

% paper title
% can use linebreaks \\ within to get better formatting as desired
\title{Non-Pipelined Versus Pipelined Implementation of Encryption Standards in
VHDL} 
\author{
  \IEEEauthorblockN{Preston Maness\\}
  \IEEEauthorblockA{Texas State University at San Marcos\\
  pmm50@txstate.edu}
}

% header
\markboth{Texas State University, Dr. Salamy, EE4321 VHDL, Spring 2014}%
{}

% make the title area
\maketitle

% Give the abstract of your lab here
\begin{abstract}
In this term paper for EE4321 VHDL with Dr. Salamy, an investigation is made 
into the advantages of pipelined architectures/dataflows through the lense of
common encryption algorithms. Two reference papers are investigated and 
analyzed, one implementing DES and the other implementing AES.
\end{abstract}

% Split your lab report into sections by calling \section{Section name}
\section{Paper Importance and Related Work}
\IEEEPARstart{T}{he} importance of encryption algorithms has grown over 
time. Traditionally, these algorithms were implemented in software and 
bottlenecked on the CPU. Thus hardware implementations were born and baked 
into modern CPUs and programmed onto FPGAs. While this was an improvement 
the throughput of these designs still left much to be desired.

These two reference papers, by X(1) and X(2), present methods of 
pipelining two common encryption algorithms. Pipelining permits the 
output of final data on every clock cycle once the pipeline is filled. 
Given the relative simplicity of these algorithms in comparison to say 
pipelined ISAs, the hazards normally encountered when pipelining are 
largely mitigated. Cheap hardware is capable of working with encrypted data
at throughputs and latencies comparable to non-encrypted data.

Related work is...

\section{Summary and Approach}

They made a block diagram of the circuit. They had a figure describing the 
encryption algos. They made non-pipelined and pipelined versions. The end.

\section{Major Contributions}

A simple introduction to pipelining with practical applications.

%\begin{tabular}{|c|c|c|c|}
%\hline
%Value & Expected & Measured & \% error \\
%\hline
%$\omega_{o}$ (Hz) & 5894 & 2747 & -53.4\\
%\hline
%phase (deg) & 45 & 41.096 & -8.67\\
%\hline
%\end{tabular}

\section{Analysis and Improvements}

Good use of block diagrams.

\section{Shortcomings}

Where is the source code? It isn't in the paper itself and isn't referenced to
exist anywhere else!

\section{Conclusion}

Good god. Please stop screen-shotting non-free software waveform viewers. Stop 
screen-shotting period! And please, PLEASE make the source code and raw data 
available and easy to find!

\bibliographystyle{ieeetran}
\bibliography{references}

\end{document}
